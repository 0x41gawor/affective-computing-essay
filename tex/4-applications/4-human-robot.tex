\subsection{Human-Robot Interaction}
Affective computing plays a crucial role in human-robot interaction (HRI), enabling robots to perceive and respond to human emotions. 
Emotionally intelligent robots can understand and adapt to users' emotional states, leading to more effective collaboration and improved user experiences. 
Applications range from social robots providing companionship to assistive robots supporting individuals with emotional needs. 
Emotion recognition allows robots to respond empathetically and appropriately in various contexts. 
For example, in healthcare settings, robots can recognize and respond to patients' emotional cues, providing comfort and emotional support. 
In customer service scenarios, emotion-aware robots can gauge customers' satisfaction levels and adjust their behavior accordingly to enhance the overall interaction.