\subsection{Mental Health and Well-being}
motion recognition systems can be used to assess individuals' emotional states, track changes over time, and provide valuable data for mental health professionals.
Mobile applications and wearables equipped with emotion tracking capabilities can enable individuals to monitor their emotional well-being, recognize patterns, 
and seek appropriate support. Virtual reality environments and simulations can also be used in exposure therapy and immersive cognitive-behavioral interventions for anxiety disorders,
 phobias, and post-traumatic stress disorder (PTSD). 
Emotion-aware virtual coaches and virtual reality-based therapies can provide personalized guidance, support, and interventions to individuals dealing with mental health challenges.