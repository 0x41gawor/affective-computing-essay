\subsection{Entertainment and Gaming}
Affective computing has made significant contributions to the entertainment industry. 
Emotion-aware gaming systems can adapt gameplay and narratives based on players' emotional states, creating more immersive and engaging experiences.
Recognized emotions can be used to adjust game difficulty, introduce personalized challenges, or dynamically change the game environment to evoke desired emotional experiences. 
Additionally, affective computing has been employed in movie recommendation systems, where emotion recognition is used to analyze viewers' emotional 
responses to movies and provide personalized recommendations based on their preferences.