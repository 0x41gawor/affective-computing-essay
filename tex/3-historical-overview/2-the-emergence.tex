\subsection{The Emergence of Affective Computing}
In the late 1990s and early 2000s, affective computing gained momentum. 
A notable project is the development of the Affectiva \cite*{affectiva} software by Rana el Kaliouby and her team. 
Affectiva employs computer vision and machine learning techniques to analyze facial expressions and recognize emotions in real-world environments.
The software has been applied to diverse domains, such as market research, driver monitoring systems, and virtual reality experiences, 
to capture and analyze users' emotional responses.

In the healthcare domain, the development of affective computing technologies has led to projects like the Virtual Interactive Presence in Augmented Reality (VIPAR) system. 
VIPAR, developed at the University of Southern California, leverages affective computing to provide emotional support for patients during medical procedures. 
The system employs a virtual human agent that can detect and respond to the patient's emotions, providing comfort and distraction during stressful situations.

Furthermore, affective computing has been instrumental in the field of robotics. 
The project "Robots with Emotions" at the University of Hertfordshire in the UK focuses 
on creating emotionally expressive robots capable of interacting with humans in socially engaging ways. 
These robots integrate various affective computing techniques, including facial expression recognition, voice analysis, and emotional behavior generation, 
to communicate and respond to human emotions effectively.