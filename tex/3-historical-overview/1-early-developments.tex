\subsection{Early Developments and Milestones}
The roots of affective computing can be traced back to the 1970s when researchers began to explore the possibility of incorporating emotional aspects into computer systems. 
One notable early development was the work of Rosalind Picard, a pioneer in the field.
In the late 1990s, Picard introduced the concept of "affective computing" and coined the term to describe the integration of emotional intelligence into technology.

Another important milestone was the development of facial expression recognition.
Paul Ekman's groundbreaking research on facial expressions in the 1970s and 1980s provided a basis for understanding the universal nature of facial expressions of emotion. 
This work laid the groundwork for the development of computer algorithms capable of recognizing and interpreting human facial expressions.

The emergence of affective computing was further fueled by advancements in physiological sensing technologies (described in section \ref{sec:physiological-measurements}).