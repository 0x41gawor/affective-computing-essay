\subsection{Machine Learning}

Machine learning algorithms are employed to train models that can recognize and classify emotions. 
These algorithms learn from labeled datasets, which consist of examples of emotions paired with corresponding features, such as facial expressions, voice patterns, or physiological signals. Through training, the models can generalize and classify new, unseen instances of emotions based on the learned patterns. 
Techniques like support vector machines, random forests, and deep learning neural networks are commonly used in affective computing to achieve accurate emotion recognition.