\subsection{Reliability and Accuracy of Emotion Recognition}
\subsubsection{Context Dependency}
Emotions are context-dependent, meaning they can be influenced by the environment, social interactions, and temporal factors. 
Affective computing systems must consider the contextual information to accurately interpret and respond to emotions. 
Incorporating context-aware models that account for environmental cues and temporal 
dynamics is necessary to improve the reliability and accuracy of emotion recognition.
\subsubsection{Ground Truth Labeling and Training Data}
Training accurate emotion recognition models requires high-quality labeled data. However, labeling emotions is inherently subjective and prone to biases. 
Consistency and agreement in ground truth labeling are critical challenges.
Researchers need to establish robust annotation protocols and guidelines to ensure reliable and consistent labeling of emotional data.
\subsubsection{Cross-Domain Generalization}
Affective computing models trained on specific datasets and contexts may not generalize well to new domains or diverse populations. 
Adapting and generalizing emotion recognition models across different domains and user groups pose challenges. 
Developing transfer learning techniques and building comprehensive and diverse datasets that cover a wide range of domains can help improve cross-domain generalization.