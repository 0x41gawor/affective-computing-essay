\subsection{Background and Definition}
Affective computing refers to the field of study and development of technologies that can recognize, interpret, and simulate human emotions.
It involves the intersection of computer science, psychology, 
and cognitive science to enable computers and artificial intelligence systems to understand and respond to human emotions effectively.

The goal of affective computing is to create machines and systems that can detect and respond to human emotions in a natural and empathetic manner. 
This involves the use of various sensors, such as cameras, microphones, and physiological sensors, to capture emotional cues such as facial expressions, 
voice tone, gestures, and physiological responses like heart rate or skin conductance.

Through the analysis of these cues, affective computing systems employ algorithms and machine learning techniques to interpret and understand the emotional state of a person. 
This understanding can then be used to provide personalized and contextually appropriate responses, whether it's in human-computer interaction, 
virtual reality, healthcare, gaming, or other applications.