%%%%%%%%%%%%%%%%%%%%%%%%%%%%%%%%%%%%%%%%%%%%%%%%%%%%%%%
%% Bachelor's & Master's Thesis Template             %%
%% Copyleft by Artur M. Brodzki & Piotr Woźniak      %%
%% Faculty of Electronics and Information Technology %%
%% Warsaw University of Technology, 2019-2020        %%
%%%%%%%%%%%%%%%%%%%%%%%%%%%%%%%%%%%%%%%%%%%%%%%%%%%%%%%

\documentclass[
    left=2.5cm,         % Sadly, generic margin parameter
    right=2.5cm,        % doesnt't work, as it is
    top=2.5cm,          % superseded by more specific
    bottom=3cm,         % left...bottom parameters.
    bindingoffset=6mm,  % Optional binding offset.
    nohyphenation=false % You may turn off hyphenation, if don't like.
]{eiti/eiti-thesis}

\langeng % Dla języka angielskiego mamy \langeng
\graphicspath{{img/}}             % Katalog z obrazkami.

\usepackage[
    backend=biber
]{biblatex}
\addbibresource{bibliografia.bib}

\begin{document}
%--------------------------------------
% Strona tytułowa
%--------------------------------------
\EngineerThesis % Dla pracy inżynierskiej mamy \EngineerThesis
\title{
    Affective Computing: Unveiling the Potential of Emotion in the Digital Age
}
\engtitle{ % Tytuł po angielsku do angielskiego streszczenia
     Affective Computing: Unveiling the Potential of Emotion in the Digital Age
}
\author{Andrzej Gawor}
\date{\the\year}


%--------------------------------------
% Streszczenie po polsku
%--------------------------------------
\cleardoublepage % Zaczynamy od nieparzystej strony
\abstract
In the realm of technological advancements, 
affective computing emerges as a groundbreaking field that aims to bridge the gap between humans and 
machines through the understanding and integration of emotions. This essay delves into the multifaceted world of affective computing, 
exploring its significance, applications, challenges, and future prospects. 
We will delve into the depths of this fascinating field, unraveling its potential to revolutionize various aspects of human-computer 
interaction and transform the way we perceive and interact with technology.

%--------------------------------------
% Spis treści
%--------------------------------------
\newpage
\tableofcontents

%--------------------------------------
% Rozdziały
%--------------------------------------
\pagestyle{headings}

\newpage
%\input{path}
\section{Introduction}
\subsection{Background and Definition}
Affective computing refers to the field of study and development of technologies that can recognize, interpret, and simulate human emotions.
It involves the intersection of computer science, psychology, 
and cognitive science to enable computers and artificial intelligence systems to understand and respond to human emotions effectively.

The goal of affective computing is to create machines and systems that can detect and respond to human emotions in a natural and empathetic manner. 
This involves the use of various sensors, such as cameras, microphones, and physiological sensors, to capture emotional cues such as facial expressions, 
voice tone, gestures, and physiological responses like heart rate or skin conductance.

Through the analysis of these cues, affective computing systems employ algorithms and machine learning techniques to interpret and understand the emotional state of a person. 
This understanding can then be used to provide personalized and contextually appropriate responses, whether it's in human-computer interaction, 
virtual reality, healthcare, gaming, or other applications.
\subsection{The Importance of Emotions in Human-Computer Interaction}
Emotions play a fundamental role in human communication and decision-making processes. 
Recognizing and responding to emotions allows computers to provide more tailored and contextually relevant support, resulting in improved \textbf{user experiences}. 
Affective computing endeavors to harness the power of emotions to create technology that is more intuitive, engaging, and empathetic.
\subsection{User Experience definition}
User experience (UX) refers to the overall experience and satisfaction that a user has when interacting with a product, system, or service. 
It encompasses the user's perceptions, emotions, and responses throughout their interaction, from the initial encounter to the final outcome. 
User experience design aims to optimize and enhance these interactions by creating intuitive, efficient, and enjoyable experiences for users.
\section{Understanding emotions}
\subsection{The science behind emotions}
Emotions are complex psychological and physiological experiences that play a crucial role in human behavior and well-being. 
The science behind emotions involves understanding the underlying neural mechanisms, physiological responses, 
and cognitive processes that contribute to emotional experiences. Let's breakdown each component.

\subsubsection{Neural mechanisms}
Emotions are rooted in the brain's complex neural network. Several brain regions, including the amygdala, 
hippocampus, prefrontal cortex, and insula, are particularly involved in emotional processing.

\begin{itemize}
    \item The amygdala, located deep within the brain, plays a vital role in the initial detection and processing of emotional stimuli. 
    It helps to evaluate the emotional significance of sensory inputs and triggers rapid emotional responses.
    \item The hippocampus is responsible for memory formation and consolidation, and it interacts with the amygdala to encode emotional memories.
    \item The prefrontal cortex, especially the ventromedial prefrontal cortex, is involved in regulating and modulating emotional responses.
    It aids in decision-making and the evaluation of potential rewards and risks.
    \item The insula is involved in the subjective experience of emotions and bodily sensations associated with emotional states.
\end{itemize}

\subsubsection{Physiological Responses}
Emotions are accompanied by physiological changes in the body. 
The autonomic nervous system (ANS) plays a significant role in mediating these responses. 
The ANS consists of the sympathetic and parasympathetic branches, which have opposing effects on bodily functions:

\begin{itemize}
    \item \textbf{Sympathetic Activation}: Emotions such as fear or anger trigger the sympathetic branch of the ANS, leading to the "fight-or-flight" response.
    This response involves increased heart rate, elevated blood pressure, rapid breathing, and the release of stress hormones like adrenaline and cortisol.
    It was a critical and highly significant reaction in a time when men had to confront predators, which were identified as a threat to our safety.
    \item \textbf{Parasympathetic Activation}: Positive emotions or relaxation trigger the parasympathetic branch, leading to a "rest-and-digest" response. 
    It results in lowered heart rate, reduced blood pressure, and a sense of calm.
\end{itemize}

\subsubsection{Cognitive Processes}
Cognitive processes refer to the mental activities involved in perceiving, interpreting, and evaluating emotional stimuli. 
These processes play a crucial role in shaping emotional experiences and determining behavioral responses.

\begin{itemize}
    \item Appraisal: Cognitive appraisal involves evaluating the meaning and significance of a given situation or stimulus. 
    It helps determine whether an event is perceived as positive, negative, or neutral, leading to corresponding emotional responses.
    \item Attention: Emotions influence our attentional focus, directing our awareness towards emotionally salient stimuli. 
    For example, feeling fear can enhance attention to potential threats in the environment.
    \item Memory and Interpretation: Emotions influence memory formation and retrieval. 
    Emotional experiences are often better remembered than neutral ones. 
    Additionally, emotions can color our interpretations of events, influencing our subsequent emotional responses.
    It is worth to know that, when you study a lot, because it can be used as a learning technique.
    \item Regulation: Emotion regulation refers to the ability to modulate emotional experiences.
    It involves cognitive processes such as reappraisal (reinterpreting the meaning of an event) and suppression (inhibiting emotional expressions). 
    These strategies can influence the intensity and duration of emotional experiences.
\end{itemize}
\subsection{Emotion recognition techniques}
To integrate emotions into computing systems, researchers have developed various techniques for recognizing and interpreting human emotions. 
These techniques range from facial expression analysis and vocal tone analysis to physiological measurements and natural language processing.

\subsubsection{The basic emotions}
Before diving into the topic of recognition, first let's define what we can recognize. In the book "Emotional Intelligence" by Daniel Goleman,
he identifies several basic emotions that are universally experienced by individuals across different cultures.
These basic emotions include:

\begin{enumerate}
    \item Happiness: Happiness is a positive emotional state characterized by feelings of joy, contentment, and satisfaction.
    It is associated with positive experiences, achievements, and pleasant circumstances.
    \item Sadness: Sadness is a negative emotional state typically associated with feelings of sorrow, grief, and melancholy. 
    It often arises in response to loss, disappointment, or separation from loved ones.
    \item Anger: Anger is a powerful and intense emotion characterized by feelings of displeasure, frustration, and hostility.
    It arises in response to perceived injustices, threats, or frustrations.
    \item Fear: Fear is an emotional response to perceived danger or threat. It triggers a heightened state of alertness and prepares the body for fight, flight, or freeze responses. 
    Fear can stem from actual physical danger or from psychological or social factors.
    \item Disgust: Disgust is an emotion that arises in response to offensive or repulsive stimuli. 
    It is associated with feelings of aversion, revulsion, and the desire to avoid or reject something distasteful.
    \item Surprise: Surprise is an emotion experienced when something unexpected or unfamiliar occurs. 
    It is characterized by a brief state of astonishment or wonder, often accompanied by physiological reactions like widened eyes or an open mouth.
    \item Contempt: Contempt is an emotion that involves feelings of scorn, disrespect, or superiority toward someone or something considered inferior or unworthy.
    It often arises from a sense of moral or social superiority.
\end{enumerate}

While these emotions are considered foundational, it is important to note that the emotional landscape is rich and nuanced, and individuals can experience
a broad spectrum of emotions that go beyond these basic categories. 
Additionally, it is worth noting that from the definition of emotion, we can often discern observable traits that are common among individuals, aiding us in identifying specific emotions.
For instance, surprise is commonly recognized by the presence of an open mouth.

\subsection{Techniques overview}

\subsubsection{Natural Language Processing}
Natural Language Processing (NLP) techniques enable computers to analyze and interpret emotions expressed through text.
Sentiment analysis and emotion classification models employ machine learning algorithms to detect emotional content in written or spoken language, such as customer reviews, 
 social media posts, and chat conversations.
By understanding the emotional tone of text, affective computing systems can provide more personalized and contextually relevant responses.
\subsubsection{Vocal Tone Analysis}
Vocal tone analysis involves extracting acoustic features from speech signals, such as pitch, intensity, and rhythm. 
Machine learning algorithms can then analyze these features to determine the emotional states expressed in speech.
Emotion recognition from vocal cues finds applications in call center analysis, voice assistants, and emotion-aware systems.
\subsubsection{Facial Expression Analysis}
Techniques such as facial action coding systems (FACS) and automated facial expression analysis algorithms utilize machine learning and 
computer vision to detect and classify facial expressions. 
These algorithms analyze factors such as muscle movements, facial landmarks, and spatial relationships to recognize emotions.
\subsubsection{Physiological Measurements}\label{sec:physiological-measurements}
Techniques for emotion recognition can leverage physiological measurements, including heart rate, skin conductance, respiration rate, and brain activity. 
Wearable devices, such as biosensors and electroencephalography (EEG) headsets, can capture these physiological signals and provide insights into the user's emotional state. 
Machine learning algorithms can then process and analyze the data to infer emotions, offering potential applications in healthcare, stress management, and affective computing research.
\subsubsection{Multimodal Fusion}
Recognizing emotions accurately often requires considering multiple modalities simultaneously. 
Multimodal fusion techniques combine information from different sources, such as facial expressions, vocal cues, and physiological signals,
to improve the robustness and accuracy of emotion recognition.
By integrating data from multiple modalities, affective computing systems can gain a more comprehensive understanding of human emotions and provide more nuanced responses.

\subsection{Challenges in emotion detection}
While significant progress has been made in emotion detection and recognition, several challenges persist in the field of affective computing.
These challenges arise due to the complex and subjective nature of emotions, as well as the diverse range of factors that influence emotional expression and interpretation. 
Let's delve into some of the key challenges faced in emotion detection:
\subsubsection{Cross-Cultural Variations}
Emotional expression and interpretation can vary across cultures. 
Different cultures may have distinct norms and display rules regarding how emotions are expressed and perceived. 
Researchers must account for cross-cultural variations by collecting data from diverse populations and developing culturally 
sensitive algorithms that consider context and cultural norms.
\subsubsection{Individual Differences}
Emotional expression and experiences are highly subjective and vary from person to person. 
Factors such as personality traits, upbringing, and life experiences can influence how individuals express and perceive emotions.
Personalization and adaptation techniques may be necessary to develop systems that can accurately capture and interpret the unique emotional patterns of individuals.
\subsubsection{Dynamic Nature of Emotions}
Emotions are dynamic and can change rapidly over time. They are influenced by contextual factors, interpersonal dynamics, and internal cognitive processes. Capturing the temporal dynamics of emotions presents a challenge for affective computing systems. 
Real-time emotion recognition and tracking systems are required to adapt to the dynamic nature of emotions and provide timely and contextually appropriate responses.
\subsubsection{Contextual Understanding}
Emotions do not occur in isolation but are influenced by contextual factors such as the environment, social interactions, and personal experiences. Understanding the context in which emotions occur is crucial for accurate emotion detection and interpretation. Affective computing systems should incorporate contextual information to improve the precision and relevance of their responses. 
Context-aware models that consider situational cues and user-specific information can enhance the overall effectiveness of emotion recognition.
\subsubsection{ Data Quality and Annotation}
Developing robust emotion recognition models requires large and diverse datasets that are accurately labeled with ground truth emotions.
However, collecting and annotating high-quality emotion data can be a challenging and time-consuming task. 
The availability of annotated datasets that cover various emotional states, cultural backgrounds,
and contextual factors is crucial for training and evaluating emotion recognition algorithms.

\section{Affective Computing: A Historical Overview}
\subsection{Early Developments and Milestones}
The roots of affective computing can be traced back to the 1970s when researchers began to explore the possibility of incorporating emotional aspects into computer systems. 
One notable early development was the work of Rosalind Picard, a pioneer in the field.
In the late 1990s, Picard introduced the concept of "affective computing" and coined the term to describe the integration of emotional intelligence into technology.

Another important milestone was the development of facial expression recognition.
Paul Ekman's groundbreaking research on facial expressions in the 1970s and 1980s provided a basis for understanding the universal nature of facial expressions of emotion. 
This work laid the groundwork for the development of computer algorithms capable of recognizing and interpreting human facial expressions.

The emergence of affective computing was further fueled by advancements in physiological sensing technologies (described in section \ref{sec:physiological-measurements}).
\subsection{The Emergence of Affective Computing}
In the late 1990s and early 2000s, affective computing gained momentum. 
A notable project is the development of the Affectiva software by Rana el Kaliouby and her team. 
Affectiva employs computer vision and machine learning techniques to analyze facial expressions and recognize emotions in real-world environments.
The software has been applied to diverse domains, such as market research, driver monitoring systems, and virtual reality experiences, 
to capture and analyze users' emotional responses.

In the healthcare domain, the development of affective computing technologies has led to projects like the Virtual Interactive Presence in Augmented Reality (VIPAR) system. 
VIPAR, developed at the University of Southern California, leverages affective computing to provide emotional support for patients during medical procedures. 
The system employs a virtual human agent that can detect and respond to the patient's emotions, providing comfort and distraction during stressful situations.

Furthermore, affective computing has been instrumental in the field of robotics. 
The project "Robots with Emotions" at the University of Hertfordshire in the UK focuses 
on creating emotionally expressive robots capable of interacting with humans in socially engaging ways. 
These robots integrate various affective computing techniques, including facial expression recognition, voice analysis, and emotional behavior generation, 
to communicate and respond to human emotions effectively.


%--------------------------------------------
% Literatura
%--------------------------------------------
\cleardoublepage % Zaczynamy od nieparzystej strony
\printbibliography

%--------------------------------------------
% Spisy 
%--------------------------------------------
% \newpage
\pagestyle{plain}

% \listoffigurestoc     % Spis rysunków.
% \vspace{1cm}          % vertical space
% \listoftablestoc      % Spis tabel.
% \vspace{1cm}          % vertical space
% \listofappendicestoc  % Spis załączników


\end{document} % Dobranoc.
